\documentclass[a4paper]{llncs}
\usepackage[T1]{fontenc}
\usepackage[utf8]{inputenc}
\usepackage{authblk}
\usepackage{graphicx, epstopdf, color, setspace, algorithm, amsfonts, amsmath, mathtools, nicefrac}
\usepackage[algo2e, noend, noline, linesnumbered]{algorithm2e}
\SetKwIF{If}{ElseIf}{Else}{if}{then}{else if}{else}{endif}
\RestyleAlgo{boxruled}
\setlength{\textfloatsep}{.5cm}
\newcommand{\func}[1]{{\sc #1}}
\newcommand{\tuple}[1]{\ensuremath{\left \langle #1 \right \rangle }}
\DontPrintSemicolon
\DeclarePairedDelimiter{\ceil}{\lceil}{\rceil}
\DeclarePairedDelimiter{\floor}{\lfloor}{\rfloor}
\newcommand{\eg}{{\it e.g.,}~}
\newcommand{\ie}{{\it i.e.,}~}
\newcommand{\bE}{\mathbb{E}}
\newcommand{\cf}{{cf.}~}
\newcommand{\TODO}[1]{\textbf{\color{red}#1}}
\graphicspath{{img/}}
\pagestyle{headings}
\title{Sequential~Halving~for~Partially~Observable~Games}

\author{Tom~Pepels\inst{1} \and Tristan~Cazenave\inst{2} \and Mark~H.M.~Winands\inst{1}}

\institute{Department of Knowledge Engineering,  Maastricht University\\ \email{\{tom.pepels,m.winands\}@maastrichtuniversity.nl} \and LAMSADE - Université Paris-Dauphine \\ \email{cazenave@lamsade.dauphine.fr}}

\begin{document}

\maketitle

\begin{abstract} 

\end{abstract}

\section{Introduction}
\label{sec:intro}

Partially observable games introduce the complexity of uncertainty in game-play. In partially observable games, some element of the game is not directly observable. Partial observability can be introduced in a game in two ways (1) by hiding certain parts of the  position to the player (e.g., hiding the rank of piece in Stratego) and (2) by introducing randomness (e.g., rolling a die in Backgammon). Other than in fully observable games, we cannot directly search for sequences of actions leading to promising moves using the partial state description.

Different approaches have been suggested for handling partial observability in Monte-Carlo Tree Search (MCTS) in such domains. Such as Determinized Monte-Carlo Search~\cite{ginsberg99} where a random game state is sampled before the search (\ie determinized), and multiple trees are maintained per determinization. The recently introduced Information Set MCTS~\cite{cowling2012} maintains information sets of states reachable in the current determinization in the tree, as such re-using statistics over multiple determinizations in a single tree.

In this paper we investigate the effects of using Sequential Halving~\cite{Karnin13SH} as a selection policy for MCTS in partially observable games. This work is a continuation of the Hybrid MCTS~\cite{Pepels14hmcts} algorithm, which was introduced as a method of minimizing simple and cumulative regret simultaneously during search.

The paper is structured as follows: first, in Section \ref{sec:mcts}, we give a broad overview of MCTS. Next, in Section \ref{sec:h-mcts} we discuss Sequential Halving, and how it may be applied to MCTS in partially observable games. After this we describe our experimental domains in Section \ref{sec:exp_dom}. Finally we show our experimental results in Section~\ref{sec:exp_res}, and conclusions in Section~\ref{sec:concl}.

\section{Monte-Carlo Tree Search}
\label{sec:mcts}

Monte-Carlo Tree Search (MCTS) is a best-first search method based on random sampling by Monte-Carlo simulations of the state space of a domain~\cite{coulom2007efficient,kocsis2006bandit}. In game play, this means that decisions are made based on the results of randomly simulated play-outs. MCTS has been successfully applied to various turn-based games such as Go~\cite{lee2010current}, Lines of Action~\cite{Winands2010b}, and Hex~\cite{arneson2010monte}. Moreover, MCTS has been used for agents playing real-time games such as the Physical Traveling Salesman~\cite{powleytsp}, real-time strategy games~\cite{balla2009uct}, and Ms~Pac-Man~\cite{realtime2014}, but also in real-life domains such as optimization, scheduling, and security~\cite{browne2012survey}.

In MCTS, a tree is built incrementally over time, which maintains statistics at each node corresponding to the rewards collected at those nodes and number of times they have been visited. The root of this tree corresponds to the current position. The basic version of MCTS consists of four steps, which are performed iteratively until a computational threshold is reached, \ie a set number of simulations, an upper limit on memory usage, or a time constraint. 

Each MCTS simulation consist of two main steps, 1) the \emph{selection} step, where moves are selected and played inside the tree according to the selection policy until a leaf is \emph{expanded}, and 2) the \emph{play-out}, in which moves are played according to a simulation policy, outside the tree. At the end of each play-out a terminal state is reached and the result is \emph{back-propagated} along the selected path in the tree from the expanded leaf to the root.

\subsection{UCT}
\label{subsec:uct}
During the selection step, a policy is required to explore the tree to decide on promising options. For this reason, the widely used Upper Confidence Bound applied to Trees (UCT)~\cite{kocsis2006bandit} was derived from the UCB1~\cite{auer2002using} policy. In UCT, each node is treated as a bandit problem whose arms are the moves that lead to different child nodes. UCT balances the exploitation of rewarding nodes whilst allowing exploration of lesser visited nodes. Consider a node $p$ with children $I(p)$, then the policy determining which child $i$ to select is defined as:

\begin{equation}
\label{eq:uct}
i^* = argmax_{i \in I(p)}\left\{ v_i + C \sqrt{ \frac{\ln{n_p}}{n_i}}\right\},
\end{equation}
where $v_i$ is the score of the child $i$ based on the average result of simulations that visited it, $n_p$ and $n_i$ are the visit counts of the current node and its child, respectively. $C$ is the exploration constant to tune. UCT is applied when the visit count of $p$ is above a threshold $T$, otherwise a child is selected at random. UCB1 and consequently, UCT incorporate both exploitation and exploration.

\subsection{MCTS in Partially Observable Games}
\label{subsec:mcts-po-games}

\section{A Hybrid PO-MCTS}
\label{sec:h-mcts}

\subsection{Sequential Halving}
\label{subsec:seq_halving}

\IncMargin{1em}
\begin{algorithm2e}[t]
\setstretch{0.95}
	\KwIn{total budget $T$, $K$ arms}
	\KwOut{recommendation $J_T$}
	\vspace{0.05cm}
	$S_0 \gets \{1,\dots,K\}$,
	$B \gets \ceil{\log_2{K}} - 1$														\;
	\BlankLine
	\For{k=0 \emph{\KwTo} $B$}{
		sample each arm $i \in S_k$, 										
		$n_k = \floor[\bigg]{\frac{T}{|S_k|\ceil{\log_2{|S|}}}}$
		times 																				\;
		\vspace{0.1cm}
		update the average reward of each arm based on the rewards 		\;
		$S_{k+1} \gets$ the $\ceil{|S_k|/2}$ arms from $S_k$ with the best average			\;
	}
	\KwRet{the single element of $S_B$}
  \caption[Sequential Halving]{Sequential Halving~\protect\cite{Karnin13SH}. \label{alg:seqhalv}}
\end{algorithm2e}
\DecMargin{1em}

\section{Experimental Domains}
\label{sec:exp_dom}

{\sc Go Fish} is a card game which is generally played by multiple players. The goal is to collect as many `books' of 4 cards of equal rank. All players hide their cards from each other, and only finished books of four cards are placed face-up on the table. Each turn, a player may ask a single other player for a specific rank. If the questioned player has any cards of the requested rank in his hand, he gives them to the requesting player, which may consequently make a new request. If the questioned player does not posses a card of the requested rank, the questioning player must `go fish', drawing a card from the stack, and the turn moves to the next player. The game ends when there are no more cards on the stack, and the player with the most finished books wins the game.

In our implementation, the game was slightly modified to allow it to be played by two players. Both players receive seven cards in hand at the start of the game. Moreover, the finished books are not similarly rewarded. Books of numbers give a score of 1, whereas books of face cards assign a score of two. As a result, when the game ends, the player with the highest score wins.

{\sc Lost Cities} is a 60-card card game, designed in 1999 by Reiner Knizia. The goal of the game is to achieve the most profitable set of expeditions to one or more of five lost cities. Players start expeditions by placing numbered cards on them, each player can start up to five expeditions regardless of the opponents' expeditions. Each card in the game has a color and a number, the colors represent one of the five expeditions, the numbers representing the score gained. Next to these cards, colored investment cards cumulatively double the score awarded for an expedition. 

Placing a card on an empty expedition `initializes' it with a cost of 20. Or, when an investment card is played, with a score of $20\times nInvest_c$, where $nInvest_c$ is the number of investment cards played on expedition $c$. These cards can only be played on an expedition when no other cards have been played on it. For example, playing the `red 5' card starts the red expedition with a cost of 20 and a score of 5 resulting in a -15 score for the player. With a single investment card on this expedition, the score will be 30. Playing more cards on the expedition leads to higher scores. However, only increasing cards may be placed on top of others. In our example, the card `red 3' can no longer be played, whereas the `red 8' card can.

Each turn, players may either play or discard a card, and draw a card from the draw pile or one of the discard piles. Discarding a card places it on top of one of the discard piles which are accessible by both players. The game ends when no card are left on the draw pile, the player with the highest score wins the game.

{\sc Phantom Domineering}
{\sc Phantom Go} 


\section{Experiments and Results}
\label{sec:exp_res}
In this section we show the results of the experiments performed on four, partially observable two-player games. PO-H-MCTS and the games were implemented in two different engines. Go Fish, Lost Cities and Phantom Domineering are implemented in a Java based engine. Phantom Go is implemented in a \emph{C++} based engine.

A uniform random selection policy is used during the play-outs, unless otherwise stated. The $C$ constant, used by UCT (Equation \ref{eq:uct}) was tuned in each game and was not re-optimized for H-MCTS, both UCT and H-MCTS use the same $C$ constant in the experiments.

\subsection{Results}
\label{subsec:results}

For each table, the results are shown with respect to the first algorithm mentioned in the captions, along with a 95\% confidence interval. For each experiment, the players' seats were swapped such that 50\% of the games are played as the first player, and 50\% as the second, to ensure no first-player or second-player bias. Because H-MCTS cannot be terminated any-time we present only results for a fixed number of simulations. In each experiment, both players are allocated a budget of both 10,000 and 25,000 play-outs.

\section{Conclusion and Future Research}
\label{sec:concl}

\bibliographystyle{splncs03}
\bibliography{h-mcts}
\end{document}
